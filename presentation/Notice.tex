\documentclass{article}

\usepackage{color}
\usepackage{graphicx}
\usepackage{tabularx}


\usepackage{geometry}
 \geometry{
 top=25mm,
 bottom=25mm,
 }


\title{Document de presentation}
\author{Justal Kevin}
\date{26/09/2015}
\renewcommand{\contentsname}{Table des matieres} 
 
\newcommand\invisiblesection[1]{%
  \refstepcounter{section}%
  \addcontentsline{toc}{section}{\protect\numberline{\thesection}#1}%
  \sectionmark{#1}} 
 
\begin{document}

\begin{center}
\textbf{\Huge{LES ANIMAUX}}
\line(1,0){300}\\
NOTICE DU JEU\\
\vspace{3cm}
\includegraphics[width=0.8\textwidth]{tablette}\\
\vspace{3cm}
\textbf{Pr\'eambule}
\end{center}

\hspace*{0.6cm}Les jeux sont un bon moyen d'am\'eliorer les compétence logico-mathématique et corporelle kinesthésique d'une personne. Mais ces effets sont décuplés sur des enfants. Poussé par leur curiosité et leur désirs d'accomplissements, ils feront souvent tout ce qui est en leur pouvoir pour arriver à la fin d'un jeu. Ces sentiments, si bien maniés, peuvent permettre d'enseigner facilement énormement de choses aux enfants qu'ils soit handicapés ou non.
\vspace{0.5cm}\\
\hspace*{0.6cm}Notre jeu met le joueur devant un problème d'association logique entre plusieurs images. En début de partie, cinq images sont donnés au joueur et 3 images en jaune sont quand à elles posés devant lui. Le joueur doit alors chercher le lien entre les images qu'il a et celle en jaune.

\newpage
\tableofcontents

\newpage
\section{Prérequis}
1. Un logiciel de décompression
\begin{itemize}
  \item Winrar
  \item Winzip (Installé de base avec windows)
\end{itemize}
2. Un navigateur web 
\begin{itemize}
  \item Internet explorer (Version minimum 9)
  \item Google Chrome (Version minimum 44)
  \item Mozilla Firefox (Version minimum 40)
  \item Safari (Version minimum 5.1)
  \item Opera (Version minimum 12)
  \item iOS (Version minimum 6.1)
  \item Android (Version minimum 2.3)
\end{itemize}

\newpage
\section{Installation}
\subsection{Décompression}
\hspace*{0.6cm}Décompresser l'archive par un simple clique droit puis extraire ici comme le montre l'image ci-dessous :\\
(Une installation de winrar sera peut-être nécessaire)
\vspace{0.5cm}\\
\fbox{
\includegraphics[width=1.0\textwidth]{winrar}\\
}
\subsection{Lancement}
Cliquer ensuite sur le dossier apparu et cliquer sur le fichier nommé "index" ou "index.html" :
\vspace{0.5cm}\\
\fbox{
\includegraphics[width=1.0\textwidth]{index}\\
}
\subsection{Plein écran}
Il est possible de mettre le jeu en plein écran en cliquant sur F11 une fois le jeu arrivé sur l'écran d'accueil. 

\newpage
\section{Gameplay}

Le jeu est décomposé en deux parties. En haut (zone1 en rouge), vous trouverez les animaux que vous pouvez déplacer et en bas (zone 2 en violet) les zones o\`u vous pouvez les déplacer.
\vspace{0.5cm}\\
\includegraphics[width=1.0\textwidth]{zone}
\vspace{0.5cm}\\
Le but du jeu est déplacer les animaux de la zone 1 vers la zone 2 en suivant une certaine logique. Reprenons notre exemple précédent et notons les animaux de la zone 1.\\
\vspace{0.5cm}\\
\includegraphics[width=1.0\textwidth]{zone1}
\vspace{0.5cm}\\
Prenons par exemple le cheval, nous allons cliquer dessus et maintenir notre clique tout en dépla\c{c}ant la souris.
\vspace{0.5cm}\\
\includegraphics[width=1.0\textwidth]{zone2}
\vspace{0.5cm}\\
Si nous rel\^achons notre clique, l'image du cheval retournera à sa place automatiquement. Maintenant, regardons avec attention la zone 2 et décrivons là aussi les images. 
\vspace{0.5cm}\\
\includegraphics[width=1.0\textwidth]{zone3}
\vspace{0.5cm}\\
Reprenons maintenant notre cheval et dépla\c{c}ont le dans la zone qui lui correspond (l'écurie). L'image se fixera alors à la zone et vous ne pourrez plus y toucher. Vous avez la bonne réponse !
\vspace{0.5cm}\\
\includegraphics[width=1.0\textwidth]{zone4}
\vspace{0.5cm}\\
Maintenant effectuons cette même opération pour chaque image restante dans la zone 1. Allez, encore un petit effort, nous y somme presque.
\vspace{0.5cm}\\
\includegraphics[width=1.0\textwidth]{zone5}
\vspace{0.5cm}\\
Si vous répondez correctement en pla\c{c}ant l'integralité des images, un message vous disant "Bravo" s'affiche alors pour vous féliciter.
\vspace{0.5cm}\\
\includegraphics[width=1.0\textwidth]{zone6}
\vspace{0.5cm}\\
Puis une nouvelle partie commence avec de nouveaux élément dans les deux zones.
\vspace{0.5cm}\\
\includegraphics[width=1.0\textwidth]{zone7}
\vspace{0.5cm}\\

\newpage
\section{évolutions}

I'm perfect, so does my game !

\newpage
\section{Outils nécessaires au développement}

\hspace*{0.6cm}Le jeu a été entièrement codé et réalisé en HTML5, CSS et JAVASCRIPT via la librairie JQUERY. Les fichiers du jeu sont lisibles et interprétés par tout les navigateurs existants en 2015.\\
\hspace*{0.6cm}Nous avons aussi utilisé LaTeX pour établir nos documents textuelles, les différents navigateurs internet pour éssayer le jeu ainsi que Photoshop pour la modification des images.\\
Pour partager le code efficacement, nous avons utilisé GIT.

\end{document}